% Created 2014-11-06 Thu 08:56
\documentclass[12pt,a4paper,koma]{article}
\usepackage[utf8]{inputenc}
\usepackage[T1]{fontenc}
\usepackage{fixltx2e}
\usepackage{graphicx}
\usepackage{longtable}
\usepackage{float}
\usepackage{wrapfig}
\usepackage[normalem]{ulem}
\usepackage{textcomp}
\usepackage{marvosym}
\usepackage{wasysym}
\usepackage{latexsym}
\usepackage{amssymb}
\usepackage{amstext}
\usepackage{hyperref}
\tolerance=1000
\usepackage{fullpage}
\author{Raymond Poling}
\date{\today}
\title{Bandit Test Writing Manual}
\hypersetup{
  pdfkeywords={},
  pdfsubject={},
  pdfcreator={Emacs 24.3.1 (Org mode 8.0)}}
\begin{document}

\maketitle
\tableofcontents

\begin{abstract}
Bandit is a simple structure for writing correctness tests. It uses a
simple, straight forward CSV style format for structuring a series of
synchronous, serial steps.  It is not the purpose of Bandit to perform
performance tests, which should be mostly asynchronous. Bandit is also
designed to be an outside observer, which is only able to observe the
actions of the system(s) under test. It a goal Bandit that except for
shell scripts, no code should be written to test a system, by using
simple generic components and simple reasoning.
\end{abstract}

\section{Introduction to Bandit}
\label{sec-1}

The format of this document is that each section following this one will
be written:
\begin{itemize}
\item Configuration: A section on the module configurations
\item Test: Actions and their arguments for each module
\end{itemize}
In that vein this portion of the document covers this information in
brief for all modules.

\subsection{Configuration}
\label{sec-1-1}

Configuration tests should be written against environments, and tests
should be as abstracted as possible from these environments. It is
not always possible, and tests have no mutable features.

Configurations are written as Clojure maps. The reason for this is to
allow for certain language features, should they be useful, such as
allowing modules to require functions to be passed in for more
elaborate capabilities.

In order for modules to be activated, and thus for their features to
be available in tests, they must minimally be represented with a
structure:
\begin{verbatim}
{ "module" {} }
\end{verbatim}
This means that the above module ``module'' has no arguments. A
better example would be this:

\begin{verbatim}
{ "rest" {}}
\end{verbatim}
As the rest module has no arguments.
\subsection{Test}
\label{sec-1-2}

All test modules have the following format:
\begin{verbatim}
identifier|module|action|argument{|arguments}*
\end{verbatim}
Where:
\begin{description}
\item[{identifier}] A string of the writer's choosing to desribe this
step of the test, intended to aid the readability of output.
\item[{module}] The module that runs this step.
\item[{action}] The action the module should perform.
\item[{argument}] All actions have at least one required argument.
\item[{arguments}] Most actions take more than one argument, and many
will take varargs.
\end{description}

Test steps are performed one after the other. A step will always be completed
when pass is displayed (although scripts may launch background processes, or
the system under test could still be processing messages from a message broker).

Tests can also contain blank lines and also any line beginning with \# is
treated as a comment.
\section{Rest Module}
\label{sec-2}
The rest module allows for testing rest interfaces using the common verbs
GET, POST, PUT, and DELETE. Each of these are defined as an action.

\subsection{Configuration}
\label{sec-2-1}

No options, to ativate:

\begin{verbatim}
{"rest" {}}
\end{verbatim}
\subsection{Test}
\label{sec-2-2}

Actions are one of:
\begin{enumerate}
\item get
\item post
\item delete
\item put
\end{enumerate}

All tests using the rest module will have rest in the second field of the
test csv. The rest module assumes responses in JSON format.

All actions have the same set of arguments:
\begin{verbatim}
url|body|code{|regex}+
\end{verbatim}

Where:
\begin{description}
\item[{url}] The url to test.
\item[{body}] A body to send with the resut query (may be empty).
\item[{code}] Expected status code (this is tested against a field within the
returned JSON body).
\item[{regex}] Regular expressions ran against the result, of which all must pass.
If regex has a ! as the first character, it will attempt to negate the match.
\end{description}
\section{Mongo Module}
\label{sec-3}
This module provides a number of features for testing or waiting for data
in MongoDB.

\subsection{Configuration}
\label{sec-3-1}
Example of all features:
\begin{verbatim}
{
  "mongo" {
    "host" "localhost"
    "port" 27017
    "writeable" true
    "db-user-pass" [{"user" "some"
      "password" "one"
      "db" "test"}]
  }
}
\end{verbatim}
Where:
\begin{description}
\item[{host}] The hostname running the mongodb that should be ran against.
\item[{port}] The port number for mongodb.
\item[{writeable}] Setting this to true (no quotes) enables destructive mongodb
operations. In case live tests are setup, this will prevent damaging
environments by allowing potentially catastrophic operations run. If
not set, or if set to anything other than \verb~true~.
\item[{db-user-pass}] The username, password, and database for authorization.
db :: Database to authenticate against.
user :: Username to authenticate with.
password :: Password to authenticate with.
\end{description}
\subsection{Test}
\label{sec-3-2}
The mongo module has various different actions for dealing with different needs
on mongodb. Only a handful are potentially destructive, and these actions are
managed by the *"writeable"* configuration setting.

All actions have the following arguments after the action argument:
\begin{verbatim}
db|collection
\end{verbatim}

Where:
\begin{description}
\item[{db}] The database to use for this action.
\item[{collection}] The collection to use for this action.
\end{description}

All following argument lists will include all arguments, for simplicity
of reading.

\subsubsection{compare Action}
\label{sec-3-2-1}

This action compares two different records (excluding certain fields) to see
if they are the same based on clojure value comparisons. Used to check if a
sequence of actions is equivalent to simply processing the last document of
a series in our testing.

Arguments are:
\begin{verbatim}
id|mongo|compare|db|collection|query1|query2{|excludes}+
\end{verbatim}

Where:
\begin{description}
\item[{query1}] A strict (keys must be wrapped in quotes) JSON query document for
a single document that has to be compared.
\item[{query2}] A strict (keys must be wrapped in quotes) JSON query document for
a single document that has to be compared.
\item[{excludes}] Fields to be excluded from the comparison. By design \_id must
be specified minimally, but any number of pipe delimited excludes may be
included.
\end{description}
\subsubsection{count Action}
\label{sec-3-2-2}
This action counts all documents in a collection and compares it to an
expected number of documents.

Arguments are:
\begin{verbatim}
id|mongo|count|db|collection|expectation
\end{verbatim}

Where:
\begin{description}
\item[{expectation}] The number of documents expected to be in the collection.
\end{description}
\subsubsection{exists Action}
\label{sec-3-2-3}
Waits for a document to appear in the document, based on a standard query.
One of the few actions that wait a set maximum time (no minimum) for an
external action to be completed.

Arguments are:
\begin{verbatim}
id|mongo|exists|db|collection|query|wait
\end{verbatim}

Where:
\begin{description}
\item[{query}] A strictly formatted JSON document to find the document.
\item[{wait}] The maximum amount of time to wait for the document to appear
in the database.
\end{description}
\subsubsection{remove Action}
\label{sec-3-2-4}
Remove all documents from a mongo db database, based on a query. Requires
*"writeable"* to be set.

Arguments are:
\begin{verbatim}
id|mongo|remove|db|collection{|queries}+
\end{verbatim}

Where:
\begin{description}
\item[{queries}] One or more pipe delimitted queries describing the documents
to remove from the collection.
\end{description}
\section{JMS Module}
\label{sec-4}
This module provides the ability to both send to and receive from JMS brokers.
Queues and topics are supported, however durable connections are not. The
publishers will publish all xml in a blank line separated file, or a set of
files in a directory. Only two real actions.

\subsection{Configuration}
\label{sec-4-1}
JMS currently only supports a single JMS broker endpoint (not queue or topic,
only the server to which one connects). As this limitation poses problems, it
is a todo to extend configuration to allow this. Such an extention should not
impact current tests.

\begin{verbatim}
{
  "jms" {
    "context" {
      "java.naming.security.principal" "admin"
      "java.naming.security.credentials" ""
      "java.naming.factory.initial" "org.apache.activemq.jndi.ActiveMQInitialContextFactory"
      "java.naming.provider.url" "tcp://localhost:61616"
      "topic.topic" "topic"
      "queue.queue" "queue"
    },
    "destinations" [{
      "destination" "topic"
      "type" "topic"
      "factory" "TopicConnectionFactory"
      "producer-consumer" "producer"
    },{
      "destination" "queue"
      "type" "queue"
      "factory" "QueueConnectionFactory"
      "producer-consumer" "producer"
  },{
      "destination" "topic"
      "type" "topic"
      "factory" "TopicConnectionFactory"
      "producer-consumer" "consumer"
    },{
      "destination" "queue"
      "type" "queue"
      "factory" "QueueConnectionFactory"
      "producer-consumer" "consumer"}]
  }
}
\end{verbatim}

Where:
\begin{description}
\item[{context}] Standard JNDI key value pairs, using the string representation of
context private static fields. Generally required is:
\begin{enumerate}
\item "java.naming.factory.initial" - Context.INITIAL\_CONTEXT\_FACTORY.
\item "java.naming.provider.url" - Context.PROVIDER\_URL.
\item "java.naming.security.principal" - Context.SECURITY\_PRINCIPAL
\item "java.naming.security.credentials" - Context.SECURITY\_CREDENTIALS
\item "topic.topic" "topic" - Represents a topic such that context lookups will
work if the resource is undefined currently on the broker for ActiveMQ.
\end{enumerate}
\item[{destinations}] A list of destinations that the test can use. In tests,
a given test destination is identified by its destination and whether it
publishes or consumes.
\item[{destination}] The destination object (name of a queue or topics).
\item[{type}] Whether this destination is a queue or a topic.
\item[{factory}] Factory object used to construct this destination.
\item[{producer-consumer}] Whether this object creates or consumes messages. Valid
values are producer or consumer, as in the key.
\end{description}
\subsection{Tests}
\label{sec-4-2}
There are only two actions in the module, publish and consume.

\subsubsection{publish Action}
\label{sec-4-2-1}

The publish action will publish all data from a file or set of files from
a directory. If more than on set of data resides within a file, it should
be separated by empty lines.

Action is:
\begin{verbatim}
id|jms|publish|destination|file-or-directory
\end{verbatim}

Where:
\begin{description}
\item[{destination}] The JMS object as defined by a destination in the config file.
\item[{file-or-directory}] The file with message body data to send, or a directory
of such files, separated by blank lines.
\end{description}
\subsubsection{consume Action}
\label{sec-4-2-2}
Allows for reading a sequend from a JMS destination, and compares it to a
set of regex of which only one need match.

Arguments are:
\begin{verbatim}
id|jms|consume|destination|number{|regex}+
\end{verbatim}

Where:
\begin{description}
\item[{destination}] A JMS destination defined in the configuration file,
as defined by the destination field.
\item[{number}] The number of messages to consume.
\item[{regex}] One or more patterns of which received messages must match at
least one.
\end{description}
\section{Shell Module}
\label{sec-5}
The shell module invokes commandline arguments and shell scripts with
arguments, allowing for use of native shell tools to support tests in live
environments.

\subsection{Configuration}
\label{sec-5-1}
No configuration, although a todo is to add a path to a predefined scripts
directory. To activate configuration must include:
\begin{verbatim}
{"shell" {}}
\end{verbatim}
\subsection{Test}
\label{sec-5-2}
The only action at present is run.

\begin{verbatim}
id|shell|run|command with arguments
\end{verbatim}

Where:
\begin{description}
\item[{command with arguments}] A command, followed by is arguments, space
separated. Environment variables cannot be used in the arguments.
\end{description}
\section{Websocket Module}
\label{sec-6}
The websocket module is a rare stateful module, that allows for the
testing of websocket services. It is stateful to prevent timeout
issues, or for messages to queue up without consumption if other
components are under test.

\subsection{Configuration}
\label{sec-6-1}
You do not need to provide an alias to construct a URL, but using
aliases means that the config file can be setup per environment, and
tests needn’t change (for tests that can be promoted through
environments). Urls should be preceded with ws://.
\begin{verbatim}
{“websocket” {“some-alias” “beginning of the url”
“another-alias” “its beginning”}}
\end{verbatim}
\subsection{Test}
\label{sec-6-2}

\subsubsection{open Action}
\label{sec-6-2-1}
Opens a websocket connection.
\begin{verbatim}
open|alias|url|handshake
\end{verbatim}
Where:
\begin{description}
\item[{alias}] used to keep track of open connections (need not be
reflected in config).
\item[{url}] is either appended to the end of the url referred to in the
config file reference for that alias, or it will be used as
is if not defined.
\item[{handshake}] is a string that will be sent over the websocket.
\end{description}
\subsubsection{receive Action}
\label{sec-6-2-2}
receive|alias|number|wait|or-pattern

Where:
\begin{description}
\item[{alias}] reference to the websocket to send messages on.
\item[{number}] The numeber of messages to receive form the websocket.
\item[{wait}] Max amount of time to wait between message receives.
\item[{or-pattern}] A set of regex patterns, one of which must match each
received message.
\end{description}
\subsubsection{send Action}
\label{sec-6-2-3}
send|alias|text

Where:
\begin{description}
\item[{alias}] reference to the websocket to send messages to.
\item[{text}] text to send through web socket.
\end{description}
\subsubsection{close Action}
\label{sec-6-2-4}
close|alias

Where:
\begin{description}
\item[{alias}] reference to the websocket connection to close.
\end{description}

\section{Kestrel Module}
\label{sec-7}
\subsection{Configuration}
\label{sec-7-1}
Kestrel uses aliases exclusively. Each alias has a map, which defines
its servers. Other options could be added to the map in the
future. Server entries must include the port. Despite the key name,
only a single server is supported at present.
\begin{verbatim}
{“kestrel” {“some-alias” {"servers" “host:port”
“another-alias” {"servers" “host:port”}}
\end{verbatim}

\subsection{Test}
\label{sec-7-2}
\subsubsection{consume Action}
\label{sec-7-2-1}
Consume text only messages from a kestrel queue.
\begin{verbatim}
consume|alias|queue|number|wait{|regex}+
\end{verbatim}
Where:
\begin{description}
\item[{alias}] used to keep track of Kestrel connections. Must exist in
config files.
\item[{queue}] the name of the queue to consume from.
\item[{number}] the number of messages to pull from Kestrel Queue.
\item[{wait}] max period of time to wait for a message to become
available on Kestrel Queue.
\item[{regex}] One or more patterns to test the text of the message
against. The patterns are logically ORed together, so that
if any one passes, they all pass.
\end{description}
\subsubsection{bin-consume Action}
\label{sec-7-2-2}
\begin{verbatim}
bin-consume|alias|queue|number|wait
\end{verbatim}
Where:
\begin{description}
\item[{alias}] used to keep track of Kestrel connections. Must exist in
config files.
\item[{queue}] the name of the queue to consume from.
\item[{number}] the number of messages to try to consume from Kestrel
queue.
\item[{wait}] how long to wait for each message to arrive at the queue.
\end{description}
\subsubsection{publish Action}
\label{sec-7-2-3}
\begin{verbatim}
publish|alias|queue|file-or-dir
\end{verbatim}
Where:
\begin{description}
\item[{alias}] used to keep track of Kestrel connections. Must exist in
config files.
\item[{queue}] the name of the queue to publish to.
\item[{file-or-directory}] load an xml file or directory of files. Uses
the same module as jms.
\end{description}
\subsubsection{bin-publish Action}
\label{sec-7-2-4}
\begin{verbatim}
bin-publish|alias|queue|file
\end{verbatim}
Where:
\begin{description}
\item[{alias}] used to keep track of Kestrel connections. Must exist in
config files.
\item[{queue}] the name of the queue to publish to.
\item[{file}] load a binary file into a Kestrel Queue. At present it will
load an entire file. Other options may present themselves later.
\end{description}
\subsubsection{peek Action}
\label{sec-7-2-5}
\begin{verbatim}
peek|alias|queue|wait{|regex}+
\end{verbatim}
Where:
\begin{description}
\item[{alias}] used to keep track of Kestrel connections. Must exist in
config files.
\item[{queue}] the name of the queue to peek at.
\item[{wait}] The period of time to wait for a message to appear on the
top of the queue.
\item[{regex}] A list of regex patterns, one of which must match the
message peeked at from the top of the queue.
\end{description}
\subsubsection{delete Action}
\label{sec-7-2-6}
\begin{verbatim}
publish|alias|queue
\end{verbatim}
Where:
\begin{description}
\item[{alias}] used to keep track of Kestrel connections. Must exist in
config files.
\item[{queue}] the name of the queue to delete.
\end{description}
\section{Running Tests}
\label{sec-8}
To run a test:
\begin{verbatim}
cd test/directory # should have the test .csv
java -jar path/to/bandit.jar bandit.core config.clj test.csv
\end{verbatim}

It is useful to write scripts to wrap up the configuration, and to move
into directories holding tests with related data.

NOTE: In lab7 you can use:

\begin{verbatim}
./check test
\end{verbatim}

This will move into the directory with the test, and run it, so all
relative paths will be related directly to the test.csv file.
\section{Sample Tests}
\label{sec-9}

\subsection{Simple Product Text}
\label{sec-9-1}
\begin{scriptsize}
\begin{verbatim}
#Clear everything, just this test
Remove all product|mongo|remove|smcdb          |product|{}
Remove journal    |mongo|remove|smcJournal     |product|{}
Remove historical |mongo|remove|smcHistoricalDb|product|{}
Remove failed     |mongo|remove|failedMessages |smcFailedMessages|{}

#publish messages to a queue
publish all xml messages|jms|publish|cmb.cibtech.na.smc_160829_isgcloud.ABS|products.xml

#Verify that all messages are consumed, pause until they are consumed
Verify exists _id|mongo|exists|smcdb|product|{"_id":"417062831"}|600
Verify exists _id|mongo|exists|smcdb|product|{"_id":"722591051"}|600
Verify exists _id|mongo|exists|smcdb|product|{"_id":"990415051"}|600
Verify exists failed ObjectId|mongo|exists|failedMessages|smcFailedMessages|{"ObjectId":"141895090"}|600
Verify exists failed ObjectId|mongo|exists|failedMessages|smcFailedMessages|{"ObjectId":"141884611"}|600
Verify exists failed ObjectId|mongo|exists|failedMessages|smcFailedMessages|{"ObjectId":"141907120"}|600

#Verify counts
Verify 9244 messages in smcdb|mongo|count|smcdb|product|9244|{}
Verify 756 messages in failedMessages|mongo|count|failedMessages|smcFailedMessages|756|{}
\end{verbatim}
\end{scriptsize}
\subsection{Failed Messages Test}
\label{sec-9-2}
\begin{scriptsize}
\begin{verbatim}
#Clear everything, just this test
Remove all product|mongo|remove|smcdb          |product|{}
Remove journal    |mongo|remove|smcJournal     |product|{}
Remove historical |mongo|remove|smcHistoricalDb|product|{}
Remove failed     |mongo|remove|failedMessages |smcFailedMessages|{}

#Backup current properties
Backup properties|shell|run|../scripts/backup-properties.sh

#Break the topology
Use historicalDB free properties|shell|run|../scripts/load-properties.sh ../properties/no-smcHistoricalDb.properties

#Start the topology
Start small-products for test|shell|run|../scripts/start-bos.sh product-small

#publish messages to a queue
Publish all xml messages|jms|publish|cmb.cibtech.na.smc_160829_isgcloud.ABS|only-good.xml

#Verify that all messages are consumed
#It is okay if some of theese fail, due to processing time in large loads
Verify exists failed ObjectId|mongo|exists|failedMessages|smcFailedMessages|{"ObjectId":"138260714"}|600
Verify exists failed ObjectId|mongo|exists|failedMessages|smcFailedMessages|{"ObjectId":"150195227"}|600
Verify exists failed ObjectId|mongo|exists|failedMessages|smcFailedMessages|{"ObjectId":"150514099"}|600

#Verify counts in failure
Verify 0 messages in db|mongo|count|smcHistoricalDb|product|0|{}
Verify 9244 messages in failedMessages|mongo|count|failedMessages|smcFailedMessages|9244|{}
Verify 9244 messages in smcdb|mongo|count|smcdb|product|9244|{}

#Stop the topology
Stop the topology|shell|run|../scripts/stop-bos.sh product-small

#Remove all of the failed messages for PRODUCT, they are known bad anyways
Remove failed PRODUCT|mongo|remove|failedMessages|smcFailedMessages|{"Type":"PRODUCT"}

#Install no smcdb properties
Install correct properties|shell|run|../scripts/load-properties.sh ../properties/no-smcdb.properties

#Restart the topology
Start small-products for test|shell|run|../scripts/start-bos.sh product-small

#publish a few more messages, so we have bad xml that's valid
Publish smcdb miss messages|jms|publish|cmb.cibtech.na.smc_160829_isgcloud.ABS|smcdb-fail.xml

#Wait for messages to stop processing
Verify exists failed ObjectId|mongo|exists|failedMessages|smcFailedMessages|{"ObjectId":"98572177","Type":"PRODUCT"}|600
Verify exists failed ObjectId|mongo|exists|failedMessages|smcFailedMessages|{"ObjectId":"12680143","Type":"P
RODUCT"}|600
Verify exists failed ObjectId|mongo|exists|failedMessages|smcFailedMessages|{"ObjectId":"150060330","Type":"P
RODUCT"}|600
Verify exists failed ObjectId|mongo|exists|failedMessages|smcFailedMessages|{"ObjectId":"150149037","Type":"P
RODUCT"}|600
Verify exists failed ObjectId|mongo|exists|failedMessages|smcFailedMessages|{"ObjectId":"150343828","Type":"P
RODUCT"}|600

#Stop the topology
Stop small products|shell|run|../scripts/stop-bos.sh product-small

#run the kestrel topology
Run Kestrel Topology|shell|run|../scripts/start-bos.sh product lab7-dl380-11

#Run the failure pump
Run smcFailedMessagePump|shell|run|../scripts/start-pump.sh SmcFailJournal

#Verify all records are pushed to db
Verify exists _id|mongo|exists|smcHistoricalDb|product|{"id":"417062831"}|600
Verify exists _id|mongo|exists|smcHistoricalDb|product|{"id":"722591051"}|600
Verify exists _id|mongo|exists|smcHistoricalDb|product|{"id":"990415051"}|600

#Verify counts are correct
#Unless we figure out how to 'fix' failedMessages to seed good xml for bad
Verify counts 9244 + 666|mongo|count|smcHistoricalDb|product|9910|{}
#Consumed messages should be removed from failed messages
Verify failed messages empty|mongo|count|failedMessages|smcFailedMessages|0|{}

#shutdown the pump
Stop pump|shell|run|../scripts/stop-pump.sh

#stop topology
Stop kestrel topology|shell|run|../scripts/stop-bos.sh product

#restore old properties
Restore old properties|shell|run|../scripts/restore-properties.sh
\end{verbatim}
\end{scriptsize}
% Emacs 24.3.1 (Org mode 8.0)
\end{document}
